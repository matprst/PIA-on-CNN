% document based on the VU Beta / BSc Thesis template

\documentclass[11pt]{article}
\usepackage{graphicx}
\usepackage{url}

      \textwidth 15cm
      \textheight 22cm
      \parindent 10pt
      \oddsidemargin 0.85cm
      \evensidemargin 0.37cm


\begin{document}

\thispagestyle{empty}

\begin{center}

Vrije Universiteit Amsterdam

\vspace{1mm}

\includegraphics[height=28mm]{vu-griffioen-white.pdf}

\vspace{1.5cm}

{\Large Bachelor Thesis}

\vspace*{1.5cm}

\rule{.9\linewidth}{.6pt}\\[0.4cm]
{\huge \bfseries Title of the Research Project\par}
{\huge \bfseries Comes Here\par}\vspace{0.4cm}
\rule{.9\linewidth}{.6pt}\\[1.5cm]

\vspace*{2mm}

{\Large
\begin{tabular}{l}
{\bf Author:} ~~Mathias Parisot ~~~~ (2618202)
\end{tabular}
}

\vspace*{1.5cm}

\begin{tabular}{ll}
{\it 1st supervisor:}   & ~~supervisor name \\
{\it 2nd supervisor:} & ~~supervisor name \\
{\it 2nd reader:}       & ~~supervisor name
\end{tabular}

\vspace*{2cm}

\textit{A thesis submitted in fulfillment of the requirements for\\ the VU Bachelor of Science degree in Computer Science }

\vspace*{1cm}

\today\\[4cm] % Date

\end{center}

\newpage

Notes:
\begin{enumerate}

    \item Curious about guidelines on conducting and on supervising research projects for young researchers? %For both students and supervisors, we recommend consulting guidelines on conducting and on supervising research projects for young researchers, respectively. 
    There are several options to learn more, for example, from this book by Sharp et al.~\cite{research:book/SharpPW02}.

    \item New to \LaTeX{}? Consult a quick guide, such as~\cite{techrep:latex,techblog:latex}.
    
    \item Q: Is this going to be very abstract? A: We use as a running example a project resulting in a Tier-1 publication~\cite{DBLP:conf/sc/AndreadisVMI18}. 
    %The project took the HP student about 25~ECs to complete, including the conference presentation and the polished, camera-ready version of the article. (Take these numbers with a grain of salt---without detracting from their achievement, the student received plenty of support from both the supervisor and the research team.)
    
    \item Q: Do I need to write a new report, if I already have a publication-ready article? A: No, that should be sufficient. You can focus on self-reflection and add that outcome as Appendix~\ref{app:selfreflection}.
    
\end{enumerate}


\section*{Important}

We do not have a lower-limit on the number of pages. Following Strunk and White's advice, we recommend keeping the page limit as low as possible. \cite{DBLP:conf/sc/AndreadisVMI18}

For the upper-limit, discuss with your supervisor on a reasonable limit, excluding the title-page and the references.




\section*{Abstract}

Explain here the context, problem, prior work, and your own approach and key results. 

Note:
\begin{enumerate}
    \item This can be seen as a short summary of the combined Introduction and Conclusion sections.
\end{enumerate}



\section{Introduction} \label{sec:introduction}

Explain the research project. Also include here the personal value you hope to derive from this project.

Explain at least:
\begin{enumerate}
    \item The context of this research project. 
    
    \item The key terms addressed in this research project. You will expand on this element throughout this work, but introduce them preferably in Section~\ref{sec:background}.
    
    \item The main problem addressed in this research project. (If possible, refer to other publications, such as vision papers or industry reports, emphasizing the importance of the problem.) 
    
    \item The key prior work related to this research project. Introduce very briefly here, as you will expand on this element in Section~\ref{sec:related}.
    
    \item The main research question, formulated according to the style of your research community. (If done in your field, also indicate the core of the approach.) You will expand on this element in Section~\ref{sec:research}.
    
    \item The main contribution of this research, for the scientific community and/or for your employer. If acceptable for your field, try to formulate as an enumeration of key contributions. You will expand on this element in Sections~\ref{sec:research} and~\ref{sec:results}.
    
    \item (If not an article) The main contribution of this research, for yourself. How did this project develop you? How did it develop your career? You will expand on this in Appendix~\ref{app:selfreflection}.
    
\end{enumerate}

For example, consider the project leading to publication~\cite{DBLP:conf/sc/AndreadisVMI18}:
\begin{enumerate}
    \item Context: datacenters, the backbone of cloud computing and our digital economy.
    \item Key terms: datacenters, scheduling, reference architecture.
    \item Problem: understanding and improving the process of scheduling in datacenters.
    \item Key prior work: research on scheduling in large-scale systems, scheduling practices in Big Tech companies (Google, Microsoft, Alibaba, etc.)
    \item Main research question: How to design a good abstraction for datacenter scheduling? Key insight: a unified reference architecture is a good  abstraction for the scheduling process.
    \item Contribution, community: a survey, a reference architecture, an analysis of existing systems as mapped to the new reference architecture, a simulator implementing the reference architecture as the scientific instrument, experiments in simulation, description of a process for others to use the reference architecture, analysis of threats to validity. 
    Plus, tangibles: a technical report accompanying the publication\footnote{The technical report is published as open science: \url{https://arxiv.org/pdf/1808.04224.pdf}}, various public talks, etc. (The team also went for and obtained the ACM reproducibility badge, which among others requires publishing FOS software and FAIR data.)
    \item Contribution, personal: development into an independent researcher.
\end{enumerate}



%\newpage

\section{Background Concepts and Models} \label{sec:background}

Explain the key concepts and models needed to understand this work.


For example, see Section~II of~\cite{DBLP:conf/sc/AndreadisVMI18}, which describes the general model of datacenter operation that the entire work is based on. Note this is not a conceptual contribution, as the model is considered common knowledge\footnote{If it was not, the authors could have either cited the original publication of the model, or claimed the model as a conceptual contribution.} in the community where the article is published. 



\section{Design of ... / Methodology for ... / Research Design on ...} \label{sec:research}

Explain the main conceptual contribution of this research, answering the research question(s) formulated in the Introduction (and, if available, in the Problem Statement report). The goal is to show that the research is feasible, important, and balanced across research questions. See, for example, Sections~I~(overview), III~(conceptual contribution: a new design), and~V.A~(conceptual contribution: a new experiment design) of~\cite{DBLP:conf/sc/AndreadisVMI18}.

Notes:
\begin{enumerate}
\item 
Describe the approach, for each research question. Explain whenever possible the method used in your approach. 

\item 
Introduce intuition about the key innovation and/or conceptual contribution. Make the article readable by a more general audience.

\item 
Try to explain why the approach should work. 

\item 
Identify and address corner-cases.

\end{enumerate}


\section{Results and Analysis} \label{sec:results}

Explain here the main results for this work. The goal is to show how the work addresses the problem, and in particular what is new or innovative.

Note:
\begin{enumerate}
    \item Different communities have different styles of presenting this material. For example, some combine results and analysis whereas others separate them, some present results per type of research method, etc. Apply what works for your research project, under the guidance of your research supervisor.
    
    \item Pay attention to the visualization of the results. Floats and tables have specific formats for each kind of analysis, so learn how to use each. Use what works for your research project, under the guidance of your research supervisor.
\end{enumerate}



\section{Discussion / Analysis of Limitations / Threats to Validity} \label{sec:discussion}

Explain in this section what you see as limitations to your contribution, or to the validity of your results.

Notes:
\begin{enumerate}
    \item Many communities include toward the end some form of discussion or analysis about the work itself. This section can have different names, see suggestions in the title of this section.
    
    \item Do not include this section without making its content valuable.
\end{enumerate}



\section{Related Work} \label{sec:related}

Explain in this section related work on the problem. The goal is to emphasize the extent and the key elements of related work. 
See also Sections~I and~VII of~\cite{DBLP:conf/sc/AndreadisVMI18}.

Notes:
\begin{enumerate}
\item 
At this stage of your research career, this part will include a brief survey of the state-of-the-art, albeit, longer than in the problem statement. This part remains guided by the project supervisor. 

\item 
Review and summarize the related work. What is known already? What should be known but isn't? What does the current work contribute?

\item 
Different communities position this part differently. We follow here advice from the systems' community Simon Peyton-Jones, about focusing on the contribution of the work in the beginning of the article, and place this Related Work section near the end of the article.

\end{enumerate}



\section{Conclusion and Future Work} \label{sec:conclusion}

Revisit the context, problem statement, related work, and research design. See, for example, Section~VIII of~\cite{DBLP:conf/sc/AndreadisVMI18}.

Explain also how you see this work developing, especially in the near-future. You could explain which limitations you'd like to address (see Section~\ref{sec:discussion}), which other frameworks or applications could leverage this work, etc.


% For more on bibliography styles, see 
% https://www.overleaf.com/learn/latex/Bibtex_bibliography_styles
\bibliographystyle{abbrv}
\bibliography{main}


\appendix

\section{Problem Statement} \label{app:problemstatement}

Include here the Problem Statement guiding the HP project.

\section{Self-Reflection} \label{app:selfreflection}

Explain in this section how this research benefited yourself. How did this project develop you? How did it develop your career? 

Also include here quantitative elements: break down the time spent on this project by type of activity / element of the project.

\end{document}
% \end{document}
